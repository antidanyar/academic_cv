\documentclass[10pt]{article}
\usepackage{fontspec}
\setmainfont{Minion 3}
\usepackage[margin=2cm]{geometry}
\usepackage{array, xcolor}
\usepackage{lipsum}
\definecolor{lightgray}{gray}{0.8}
\newcolumntype{L}{>{\raggedleft}p{0.14\textwidth}}
\newcolumntype{R}{p{0.8\textwidth}}
\newcommand\VRule{\color{lightgray}\vrule width 0.5pt}

\title{\Huge Daniar Kasenov}
\author{antidanyar@protonmail.com\\mortypines@gmail.com}
\date{Last update: December 20th, 2022}
\begin{document}
\maketitle

\vspace{-3em}

\section*{Academic interests}
\begin{tabular}{L!{\VRule}R}
{Broad}&{\bf Generative linguistics, syntax and its interfaces, modular approaches to linguistic theory}\\
{Narrow}&{Allosemy, Nanosyntax, DM, Turkic languages, Russian}\\
\end{tabular}

\section*{Work Experience}
\begin{tabular}{L!{\VRule}R}
{2020--today}&{\bf Research assistant at HSE Laboratory of Formal Models in Linguistics}\\
{}&{Research, organizing seminars for presenting students' research}\\
{Spring 2023}&{Teaching assistant in formal semantics for 2nd and 3rd year linguistics majors}\\
{Spring 2022} & {Teaching assistant in syntax for 2nd year linguistics majors} \\
{Spring 2021}&{Teaching assistant in computer science for 1st year linguistics majors}\\[5pt]
\end{tabular}

\section*{Education}
\begin{tabular}{L!{\VRule}R}
2019--2023&{\bf BA Program in Fundamental and Computational Linguistics} (GPA 8.87/10.0)\\
{}&{Higher School of Economics, Moscow, Russia} \\
{Thesis title} & { The role of syntactic context in modal semantics in non-standard modal expressions} \\
{} & {Supervisor: Alexander Podobryaev}\\
\end{tabular}

\section*{Publications}
\begin{tabular}{L!{\VRule}R}
{In prep.} & {Particle nature of Turkic evidential perfect in Terek Kumyk (in Russian)}\\
{} & {w/ Daria Paramonova (MSU)}\\
{In prep.} & {Third person sensitive accusative case allomorphy in Balkar and Kumyk}\\
{To appear} & {F=PL syncretism and default agreement: case of Shughni}\\
{} & {\it Typology of Morphosyntactic Parameters, 5(2)}\\
{} & {w/ Alexander Sergienko and Artyom Badeev (NRU HSE)}\\
{Under review} & {Multiple exponence in Russian exhortatives: not what it seems}\\
{To appear} & {Balkar X-marking: a change in progress}\\
{} & {\it Proceedings of Tu+7}\\
{To appear} & {Semantic person agreement: a view from Balkar (in Russian)}\\
{} & {\it Proceedings of Small Languages in Big Linguistics 2022}\\
{2021} & {Egophoricity as interpretable agreement} \\
{} & {\it Typology of Morphosyntactic Parameters, 4(2)}
\end{tabular}

\section*{Grants}
\begin{tabular}{L!{\VRule}R}
{2022--2024} & {\bf Scalarity in grammar and lexicon: a semantic-typological approach}\\
{} & {Supported by Russian Scientific Foundation (\# 22-18-00285)}\\
{} & {PI: Sergey Tatevosov}\\
{2021--2023} & {\bf Morphology of agreement}\\
{} & {Supported by Russian Foundation for Basic Research (\# 20-512-26004)} \\
{} & {PI: Svetlana Toldova} \\
\end{tabular}


\section*{Conference talks}
\begin{tabular}{L!{\VRule}R}
{Feb 2023} & {\it ConSOLE 31}\\
{} & {Non-finite clauses and root modality: a view from Russian}\\
{Dec 2022} & {\it 7th Asian Junior Linguists conference (AJL7)}\\
{} & {Deriving sufficiency with strengthening in conditionals: case of Terek Kumyk}\\
{} & {w/ Daria Paramonova, Anastasiia Krainova (MSU)}\\
{Nov 2022} & {\it The 2nd Conference on Uralic, Altaic and Paleo-Asiatic languages}\\
{} & {Particle nature of Turkic evidential perfect in Terek Kumyk (in Russian)}\\
{} & {w/ Daria Paramonova (MSU)}\\
{Nov 2022} & {\it The 2nd Conference on Uralic, Altaic and Paleo-Asiatic languages}\\
{} & {Syntactic variation in complement clauses of Terek Kumyk modal \textit{jaraj} as the consequence of their subject status (in Russian)}\\
{} & {w/ Anastasiia Krainova (MSU)}\\
{Nov 2022} & {\it 19th Conference on Typology and Grammar for Young Scholars} \\
{} & {Sufficiency by antecedent strengthening in Terek Kumyk}\\
{} & {w/ Daria Paramonova, Anastasiia Krainova (MSU)}\\
{Nov 2022} & {\it 19th Conference on Typology and Grammar for Young Scholars} \\
{} & {Allomorphy of the plural affix in Terek Kumyk: avoiding ABA phonologically}\\
{Oct 2022} & {\it Typology of Morphosyntactic Parameters 12}\\
{} & {F=PL syncretism and default gender: case of Shughni}\\
{} & {w/ Alexander Sergienko, Artyom Badeev (NRU HSE)}\\
{Oct 2022} & {\it Formal Description of Slavic Languages 15 (FDSL 15)}\\
{} & {Accounting for Russian superlatives with Nanosyntax}\\
{Oct 2022} & {\it Formal Description of Slavic Languages 15 (FDSL 15)} \\
{} & {Stress shift in Russian prepositional phrases: a strict CV approach}\\
{} & {w/ Alexandra Shikunova (NRU HSE)}\\
{Apr 2022} & {\it Small Languages in Big Linguistics} \\
{} & {Semi-direct speech and semantic agreement: view from Balkar (in Russian)} \\
{Mar 2022} & {\it Theoretical Linguistics at Keio: Semantics Conference} \\
{} & {Imperatives as counterfactual antecedents in Russian: a stripped down approach} \\
{Feb 2022} & {\it Tu+7}\\
{} & {Balkar conditional marker as antecedent X-marking} \\
{Dec 2021} & {\it Multiple Exponence workshop @ ZAS}\\
{ } & {Multiple exponence in Russian exhortatives: Not what it seems} \\
{Nov 2021} & {\it 18th Conference on Typology and Grammar for Young Scholars} \\
{ } & {Indexical shift and monstrous agreement in Balkar} \\
{Nov 2021} & {\it 18th Conference on Typology and Grammar for Young Scholars} \\
{ } & {Russian {\it imenno}: A diachrony}\\ 
{} & {w/ Alexandra Konovalova, Alexey Kozlov, and Maxim Bazhukov (NRU HSE)} \\
{Oct 2021} & {\it Typology of Morphosyntactic Parameters 11} \\
{} & {Egophoricity: a set-based \textsc{agree} analysis} \\
{Sep 2021} & {\it Slavic Linguistic Society 16 (SLS16)} \\
{} & {Russian imperative as a counterfactual antecedent}
\end{tabular}

\section*{Other achievements}
\begin{tabular}{L!{\VRule}R}
{2022} & {\bf Honorable mention in HSE Student Research Paper Competition}\\
{} & {Multiple exponence in Russian exhortatives: not what it seems}\\
{2021} & {\bf Honorable mention in HSE Student Resarch Paper Competition}\\
{} & {Morphosyntax of Mehweb agreement, supervised by Pavel Rudnev}\\
\end{tabular}


\section*{Additional education}
\begin{tabular}{L!{\VRule}R}
{2023}&{\it Winter school of HSE and KFU ``Qualitative and quantitative approaches to language variation''}\\
{}&{Student talk: {*}ABA generalizations in modern morphosyntax}\\
{2022}&{\it Summer school in Logic and Formal Philosophy of LLFP HSE}\\
{}&{Student talk: Linguistic arguments against Kratzer's semantics for modality (in Russian)}\\
2022&{\it Tomsk National University's Summer School on Philosophy of Language and Logic}\\
{}&{Student talk: Towards scalar semantics of quantifiers (in Russian)}\\
2022&{\it V-NYI Summer School in Linguistics, Cognitive and Cultural Studes}\\
2022&{\it W-NYI Winter School in Linguistics, Cognitive and Cultural Studes}\\
2021&{\it HSE International Laboratory for Logic, Linguistics and Formal Philosophy's 1st Summer School} \\
2021&{\it MCCS Summer School `Possible Worlds Metaphysics'} \\
2021&{\it W-NYI Winter School in Linguistics, Cognitive and Cultural Studes} \\
2020&{\it Intensional semantics, HSE Laboratory of Formal Models in Linguistics}\\
{}&{taught by Alexander Podobryaev and Natalia Ivlieva}\\
2020&{\it V-NYI Summer School in Linguistics, Cognitive and Cultural Studies}\\
2020&{{\it MIT 24.899: Topics in Linguistics and Philosophy}: course on conditionals} \\
{}&{taught by Sabine Iatridou, Kai von Fintel, Justin Khoo}\\
\end{tabular}

\section*{Fieldwork}
\begin{tabular}{L!{\VRule}R}
{August 2022} & {{\bf Kumyk} (Turkic)}\\
{} & {As a member of the Lomonosov Moscow State University Altaic expedition}\\
{}&{Supervisor: Sergey Tatevosov}\\
{August 2021} & {{\bf Balkar} (Turkic)}\\
{} & {As a member of the Lomonosov Moscow State University Altaic expedition}\\
{}&{Supervisor: Sergey Tatevosov}\\
\end{tabular}


\section*{Languages}
\begin{tabular}{L!{\VRule}R}
{\bf English}&{\bf Fluent}\\
{\bf German}&{\bf Intermediate}\\
Latin&Reading with a dictionary\\
Greek&Reading with a dictionary\\
Finnish&Pre-intermediate\\
Russian&Native\\
\end{tabular}

\section*{Programming skills}
\begin{tabular}{L!{\VRule}R}
Python&Intermediate\\
\LaTeX & Intermediate\\
\end{tabular}

\end{document}












