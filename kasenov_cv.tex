\documentclass[10pt]{article}
\usepackage{fontspec}
\setmainfont{Minion 3}
\usepackage[margin=2cm]{geometry}
\usepackage{array, xcolor}
\usepackage{longtable}
\usepackage{lipsum}
\definecolor{lightgray}{gray}{0.8}
\newcolumntype{L}{>{\raggedleft}p{0.14\textwidth}}
\newcolumntype{R}{p{0.8\textwidth}}
\newcommand\VRule{\color{lightgray}\vrule width 0.5pt}

\usepackage{titling}
\pretitle{\begin{flushleft}\LARGE}
\posttitle{\par\end{flushleft}\vskip 0.5em}
\preauthor{\begin{flushleft}\large}
\postauthor{\par\end{flushleft}}
\predate{\begin{flushleft}\large}
\postdate{\par\end{flushleft}}

\title{\Huge Daniar Kasenov}
\author{Contact: antidanyar@protonmail.com ; mortypines@gmail.com}
\date{Last update: May 30th, 2023}
\begin{document}
\maketitle

\vspace{-3em}

\subsection*{Academic interests}
\begin{tabular}{L!{\VRule}R}
{Broad}&{\bf Generative linguistics, syntax and its interfaces, modular approaches to linguistic theory}\\
{Narrow}&{Allosemy, allomorphy, modality, Turkic languages, Russian}\\
\end{tabular}

\subsection*{Work Experience}
\begin{tabular}{L!{\VRule}R}
{2020--today}&{\bf Research assistant at HSE Laboratory of Formal Models in Linguistics}\\
{}&{Research, organizing seminars for presenting students' research}\\
{2023-2024}&{Instructor in syntax for 2nd year linguistics majors at HSE}\\
{Spring 2023}&{TA in formal semantics for 2nd and 3rd year linguistics majors at HSE}\\
{Spring 2022} & {TA in syntax for 2nd year linguistics majors at HSE} \\
{Spring 2021}&{TA in computer science for 1st year linguistics majors at HSE}\\[5pt]
\end{tabular}

\subsection*{Education}
\begin{tabular}{L!{\VRule}R}
2019--2023&{\bf BA Program in Fundamental and Computational Linguistics} (GPA 8.87/10.0)\\
{}&{Higher School of Economics, Moscow, Russia} \\
{Thesis title} & { The role of syntactic context in modal semantics in non-standard modal expressions} \\
{} & {Supervisor: Alexander Podobryaev}\\
\end{tabular}

\subsection*{Publications}
\begin{tabular}{L!{\VRule}R}
{In prep.} & {Non-finite clauses and root modality: a view from Russian}\\
{} & {\it Intended for proceedings of ConSOLE31}\\
{In prep.} & {Accounting for Russian superlatives with Nanosyntax}\\
{} & {\it Intended for proceedings of FDSL15}\\
{In prep.} & {Russian prepositional phrases on the syntax-prosody interface}\\
{} & {\it Intended for proceedings of FDSL15}\\
{} & {w/ Alexandra Shikunova (NRU HSE)}\\
{In prep.} & {Particle nature of Turkic evidential perfect in Terek Kumyk (in Russian)}\\
{} & {w/ Daria Paramonova (MSU)}\\
{Under review} & {Multiple exponence in Russian exhortatives: not what it seems}\\
{2023} & {Balkar X-marking: a change in progress}\\
{} & {\it Proceedings of Tu+7}\\
{2023} & {Third person sensitive accusative case allomorphy in Balkar and Kumyk}\\
{} & {\it In S. Toldova, P. Caha, P. Rudnev (eds.) Many facets of agreement. LINCOM Europe }\\
{2022} & {F=PL syncretism and default agreement: case of Shughni}\\
{} & {\it Typology of Morphosyntactic Parameters, 5(2)}\\
{} & {w/ Alexander Sergienko and Artyom Badeev (NRU HSE)}\\
{2022} & {Semantic person agreement: a view from Balkar (in Russian)}\\
{} & {\it Proceedings of Small Languages in Big Linguistics 2022}\\
{2021} & {Egophoricity as interpretable agreement} \\
{} & {\it Typology of Morphosyntactic Parameters, 4(2)}
\end{tabular}

\subsection*{Conference talks}
\begin{longtable}{L!{\VRule}R}
{Oct 2023}&{\it Typology of Morphosyntactic Parameters 13}\\
{} & {Russian \textit{to}-conditionals as hanging topic constructions}\\
{} & {w/ Daria Paramonova (MSU)}\\
{Oct 2023}&{\it Typology of Morphosyntactic Parameters 13}\\
{} & {Verb \textit{dopuskat'}, Neg-Raising, and the nature of scalar alternatives (in Russian)}\\
{} & {w/ Daria Sidorkina (HSE)}\\
{May 2023}&{\it RALFE 2023}\\
{} & {The place of modal ambiguity in the modular grammar}\\
{May 2023} & {\it WCCFL 41}\\
{} & {Revisiting Basque xe-comparatives: in support of structural adjacency}\\
{May 2023} & {\it 12th American Phonology Conference}\\
{} & {Power of direction in Precedence-based (morpho-)phonology: ABA in Terek Kumyk}\\
{May 2023} & {\it 12th American Phonology Conference}\\
{} & {Russian iotation: length is key (second author)}\\
{} & {w/ Alexandra Shikunova (HSE University)}\\
{Apr 2023} & {\it Small Languages in Big Linguistics}\\
{} & {Two paradigms of Balkar agreement and the shape of Nanosyntactic trees}\\
{Apr 2023} & {\it Small Languages in Big Linguistics} \\
{} & {Syntactic correlates of dynamic modality in Terek Kumyk (in Russian)} \\
{} & {w/ Daria Paramonova (MSU)}\\
{Apr 2023} & {\it NACIL 3}\\
{} & {Shughni comples predicates and denominal verbs (second author)}\\
{} & {w/ Alexander Sergienko (HSE university)}\\
{Apr 2023} & {\it Workshop on Modality in Understudied Languages}\\
{} & {Variation of dynamic modals on the syntax-semantics interface: case of Terek Kumyk}\\
{} & {w/ Daria Paramonova (MSU)}\\
{Apr 2023} & {\it FASL 2023 (alternate list)}\\
{} & {Syntactic Neg-Raising is untenable: evidence from Russian Negative Concord}\\
{Feb 2023} & {\it ConSOLE 31}\\
{} & {Non-finite clauses and root modality: a view from Russian}\\
{Dec 2022} & {\it 7th Asian Junior Linguists conference (AJL7)}\\
{} & {Deriving sufficiency with strengthening in conditionals: case of Terek Kumyk}\\
{} & {w/ Daria Paramonova, Anastasiia Krainova (MSU)}\\
{Nov 2022} & {\it The 2nd Conference on Uralic, Altaic and Paleo-Asiatic languages}\\
{} & {Particle nature of Turkic evidential perfect in Terek Kumyk (in Russian)}\\
{} & {w/ Daria Paramonova (MSU)}\\
{Nov 2022} & {\it The 2nd Conference on Uralic, Altaic and Paleo-Asiatic languages}\\
{} & {Syntactic variation in complement clauses of Terek Kumyk modal \textit{jaraj} as the consequence of their subject status (in Russian)}\\
{} & {w/ Anastasiia Krainova (MSU)}\\
{Nov 2022} & {\it 19th Conference on Typology and Grammar for Young Scholars} \\
{} & {Sufficiency by antecedent strengthening in Terek Kumyk}\\
{} & {w/ Daria Paramonova, Anastasiia Krainova (MSU)}\\
{Nov 2022} & {\it 19th Conference on Typology and Grammar for Young Scholars} \\
{} & {Allomorphy of the plural affix in Terek Kumyk: avoiding ABA phonologically}\\
{Oct 2022} & {\it Typology of Morphosyntactic Parameters 12}\\
{} & {F=PL syncretism and default gender: case of Shughni}\\
{} & {w/ Alexander Sergienko, Artyom Badeev (NRU HSE)}\\
{Oct 2022} & {\it Formal Description of Slavic Languages 15 (FDSL 15)}\\
{} & {Accounting for Russian superlatives with Nanosyntax}\\
{Oct 2022} & {\it Formal Description of Slavic Languages 15 (FDSL 15)} \\
{} & {Stress shift in Russian prepositional phrases: a strict CV approach}\\
{} & {w/ Alexandra Shikunova (HSE University)}\\
{Apr 2022} & {\it Small Languages in Big Linguistics} \\
{} & {Semi-direct speech and semantic agreement: view from Balkar (in Russian)} \\
{Mar 2022} & {\it Theoretical Linguistics at Keio: Semantics Conference} \\
{} & {Imperatives as counterfactual antecedents in Russian: a stripped down approach} \\
{Feb 2022} & {\it Tu+7}\\
{} & {Balkar conditional marker as antecedent X-marking} \\
{} & {}\\
{Dec 2021} & {\it Multiple Exponence workshop @ ZAS}\\
{ } & {Multiple exponence in Russian exhortatives: Not what it seems} \\
{Nov 2021} & {\it 18th Conference on Typology and Grammar for Young Scholars} \\
{ } & {Indexical shift and monstrous agreement in Balkar} \\
{Nov 2021} & {\it 18th Conference on Typology and Grammar for Young Scholars} \\
{ } & {Russian {\it imenno}: A diachrony}\\ 
{} & {w/ Alexandra Konovalova, Alexey Kozlov, and Maxim Bazhukov (HSE university)} \\
{Oct 2021} & {\it Typology of Morphosyntactic Parameters 11} \\
{} & {Egophoricity: a set-based \textsc{agree} analysis} \\
{Sep 2021} & {\it Slavic Linguistic Society 16 (SLS16)} \\
{} & {Russian imperative as a counterfactual antecedent}
\end{longtable}


\subsection*{Grants}
\begin{tabular}{L!{\VRule}R}
{2022--2024} & {\bf Scalarity in grammar and lexicon: a semantic-typological approach}\\
{} & {Supported by Russian Scientific Foundation (\# 22-18-00285)}\\
{} & {PI: Sergey Tatevosov}\\
{2021--2023} & {\bf Morphology of agreement}\\
{} & {Supported by Russian Foundation for Basic Research (\# 20-512-26004)} \\
{} & {PI: Svetlana Toldova} \\
\end{tabular}

\subsection*{Other achievements}
\begin{tabular}{L!{\VRule}R}
{2023} & {\bf Grant for CreteLing summer school (300€ fee waiver)}\\
{2022} & {\bf Honorable mention in HSE Student Research Paper Competition}\\
{} & {Multiple exponence in Russian exhortatives: not what it seems}\\
{2021} & {\bf Honorable mention in HSE Student Resarch Paper Competition}\\
{} & {Morphosyntax of Mehweb agreement, supervised by Pavel Rudnev}\\
\end{tabular}


\subsection*{Additional education}
\begin{tabular}{L!{\VRule}R}
{2023} & {\it CreteLing 2023}\\
{} & {Advanced courses on phonology, morphosyntax, and semantics}\\
{2023}&{\it W-NYI Winter School in Linguistics, Cognitive and Cultural Studes}\\
{} & {Advanced courses on phonology and semantics}\\
{2022}&{\it Winter school of HSE and KFU ``Qualitative and quantitative approaches to language variation''}\\
{}&{Student talk: {*}ABA generalizations in modern morphosyntax}\\
{2022}&{\it Summer school in Logic and Formal Philosophy of LLFP HSE}\\
{}&{Student talk: Linguistic arguments against Kratzer's semantics for modality (in Russian)}\\
2022&{\it Tomsk National University's Summer School on Philosophy of Language and Logic}\\
{}&{Student talk: Towards scalar semantics of quantifiers (in Russian)}\\
2022&{\it V-NYI Summer School in Linguistics, Cognitive and Cultural Studes}\\
{} & {Advanced courses on phonology and semantics}\\
2022&{\it W-NYI Winter School in Linguistics, Cognitive and Cultural Studes}\\
{} & {Advanced courses on phonology and morphosyntax}\\
{2021}&{\it Summer school in Logic and Formal Philosophy of LLFP HSE}\\
2021&{\it MCCS Summer School `Possible Worlds Metaphysics'} \\
2021&{\it W-NYI Winter School in Linguistics, Cognitive and Cultural Studes} \\
{} & {Advanced courses on morphosyntax and semantics}\\
2020&{\it Intensional semantics, HSE Laboratory of Formal Models in Linguistics}\\
{}&{taught by Alexander Podobryaev and Natalia Ivlieva}\\
2020&{\it V-NYI Summer School in Linguistics, Cognitive and Cultural Studies}\\
{} & {Advanced courses on morphosyntax and semantics}\\
\end{tabular}

\subsection*{Fieldwork experience}
\begin{tabular}{L!{\VRule}R}
{August 2022} & {{\bf Terek Kumyk} (Turkic)}\\
{} & {As a member of the Lomonosov Moscow State University Altaic expedition}\\
{}&{Supervisors: Sergey Tatevosov, Petr Rossyaykin}\\
{August 2021} & {{\bf Balkar} (Turkic)}\\
{} & {As a member of the Lomonosov Moscow State University Altaic expedition}\\
{}&{Supervisors: Sergey Tatevosov, Alexander Podobryaev}\\
\end{tabular}

\subsection*{References}
\begin{tabular}{l!{\VRule}l}
Alexander Podobryaev & sasha.podobryaev@gmail.com\\
Pavel Rudnev & pasha.rudnev@gmail.com\\
Sergey Tatevosov & tatevosov@gmail.com\\
Svetlana Toldova&toldova@yandex.ru\\
\end{tabular}


\end{document}












