\documentclass[10pt]{article}
\usepackage{fontspec}
\setmainfont{Minion 3}
\usepackage[margin=2cm]{geometry}
\usepackage{array, xcolor}
\usepackage{lipsum}
\definecolor{lightgray}{gray}{0.8}
\newcolumntype{L}{>{\raggedleft}p{0.14\textwidth}}
\newcolumntype{R}{p{0.8\textwidth}}
\newcommand\VRule{\color{lightgray}\vrule width 0.5pt}

\usepackage{titling}
\pretitle{\begin{flushleft}\LARGE}
\posttitle{\par\end{flushleft}\vskip 0.5em}
\preauthor{\begin{flushleft}\large}
\postauthor{\par\end{flushleft}}
\predate{\begin{flushleft}\large}
\postdate{\par\end{flushleft}}

\title{\Huge Данияр Касенов}
\author{E-mail: antidanyar@protonmail.com ; mortypines@gmail.com}
\date{Последнее обновление: 15 янв 2023}
\begin{document}
\maketitle

\vspace{-3em}

\subsection*{Академические интересы}
\begin{tabular}{L!{\VRule}R}
{Широко}&{\bf Генеративная лингвистика, синтаксис и интерфейсы, модулярные подходы к лингвистике}\\
{Узко}&{Аллосемия, алломорфия, модальность, тюркские языка, русский язык}\\
\end{tabular}

\subsection*{Опыт работы}
\begin{tabular}{L!{\VRule}R}
{2020--}&{\bf Стажёр-исследователь в НУЛ по формальным моделям в лингвистике ВШЭ}\\
{}&{Проведение исследований, организация студенческих семинаров}\\
{Весна 2023}&{Учебный ассистент по курсу Формальная семантика}\\
{Весна 2022} & {Учебный ассистент по курсу Теория языка (синтаксис)} \\
{Весна 2021}&{Учебный ассистент по курсу Программирование и лингвистические данные}\\[5pt]
\end{tabular}

\subsection*{Образование}
\begin{tabular}{L!{\VRule}R}
2019--2023&{\bf ОП Фундаментальная и Компьютерная Лингвистике} (GPA 8.87/10.0)\\
{}&{НИУ ВШЭ, Москва, Россия} \\
\end{tabular}

\subsection*{Публикации}
\begin{tabular}{L!{\VRule}R}
{Рукопись} & {Accounting for Russian superlatives with Nanosyntax}\\
{} & {\it Подготовлена для Proceedings of FDSL15}\\
{Рукопись} & {Russian prepositional phrases on the syntax-prosody interface}\\
{} & {\it Подготовлена для Proceedings of FDSL15}\\
{} & {совместно с Александрой Шикуновой}\\
{На рецензии} & {Multiple exponence in Russian exhortatives: not what it seems}\\
{2023} & {Balkar X-marking: a change in progress}\\
{} & {\it Proceedings of Tu+7}\\
{2023} & {Third person sensitive accusative case allomorphy in Balkar and Kumyk}\\
{} & {\it In S. Toldova, P. Caha, P. Rudnev (eds.) Many facets of agreement. LINCOM Europe }\\
{2022} & {F=PL syncretism and default agreement: case of Shughni}\\
{} & {\it Typology of Morphosyntactic Parameters, 5(2)}\\
{} & {совместно с Александром Сергиенко и Артемом Бадеевым}\\
{2022} & {Семантическое согласование по лицу: взгляд со стороны балкарского языка}\\
{} & {\it Малые языки в большой лингвистике 2022}\\
{2021} & {Egophoricity as interpretable agreement} \\
{} & {\it Typology of Morphosyntactic Parameters, 4(2)}
\end{tabular}

\subsection*{Гранты}
\begin{tabular}{L!{\VRule}R}
{2022--2024} & {\bf Скалярность в грамматике и лексиконе: семантико-типологическихй подход}\\
{} & {РНФ \# 22-18-00285}\\
{2021--2023} & {\bf Morphology of agreement}\\
{} & {РФФИ \# 20-512-26004} \\
\end{tabular}


\subsection*{Выступления на конференциях}
\begin{tabular}{L!{\VRule}R}
{02-2023} & {\it ConSOLE 31}\\
{} & {Non-finite clauses and root modality: a view from Russian}\\
{12-2022} & {\it 7th Asian Junior Linguists conference (AJL7)}\\
{} & {Deriving sufficiency with strengthening in conditionals: case of Terek Kumyk}\\
{} & {w/ Daria Paramonova, Anastasiia Krainova (MSU)}\\
{11-2022} & {\it Вторая конференция по уральским, алтайским и палеоазиатским языкам}\\
{} & {Частичная судьба тюркского эвиденциального перфекта в терском кумыкском}\\
{} & {совместно с Дарьей Парамоновой}\\
{11-2022} & {\it Вторая конференция по уральским, алтайским и палеоазиатским языкам}\\
{} & {Синтаксическая вариативность модального предиката {\it ярай} в терском кумыкском}\\
{} & {совместно с Анастасией Крайновой}\\
{11-2022} & {\it 19th Conference on Typology and Grammar for Young Scholars} \\
{} & {Sufficiency by antecedent strengthening in Terek Kumyk}\\
{} & {совместно с Дарьей Парамоновой и Анастасией Крайновой}\\
{11-2022} & {\it 19th Conference on Typology and Grammar for Young Scholars} \\
{} & {Allomorphy of the plural affix in Terek Kumyk: avoiding ABA phonologically}\\
{10-2022} & {\it Typology of Morphosyntactic Parameters 12}\\
{} & {F=PL syncretism and default gender: case of Shughni}\\
{} & {совместно с Александром Сергиенко и Артемом Бадеевым}\\
{10-2022} & {\it Formal Description of Slavic Languages 15 (FDSL 15)}\\
{} & {Accounting for Russian superlatives with Nanosyntax}\\
{10-2022} & {\it Formal Description of Slavic Languages 15 (FDSL 15)} \\
{} & {Stress shift in Russian prepositional phrases: a strict CV approach}\\
{} & {совместно с Александрой Шикуновой}\\
{04-2022} & {\it Малые языки в Большой Лингвистике} \\
{} & {Полукосвенная речь и семантическое согласование: взгляд со стороны балкарского языка} \\
{03-2022} & {\it Theoretical Linguistics at Keio: Semantics Conference} \\
{} & {Imperatives as counterfactual antecedents in Russian: a stripped down approach} \\
{02-2022} & {\it Tu+7}\\
{} & {Balkar conditional marker as antecedent X-marking} \\
{12-2021} & {\it Multiple Exponence workshop @ ZAS}\\
{ } & {Multiple exponence in Russian exhortatives: Not what it seems} \\
{11-2021} & {\it 18th Conference on Typology and Grammar for Young Scholars} \\
{ } & {Indexical shift and monstrous agreement in Balkar} \\
{11-2021} & {\it 18th Conference on Typology and Grammar for Young Scholars} \\
{ } & {Russian {\it imenno}: A diachrony}\\ 
{} & {w/ Alexandra Konovalova, Alexey Kozlov, and Maxim Bazhukov (NRU HSE)} \\
{10-2021} & {\it Typology of Morphosyntactic Parameters 11} \\
{} & {Egophoricity: a set-based \textsc{agree} analysis} \\
{09-2021} & {\it Slavic Linguistic Society 16 (SLS16)} \\
{} & {Russian imperative as a counterfactual antecedent}
\end{tabular}

\end{document}












